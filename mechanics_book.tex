\documentclass[12pt]{book}
\usepackage{amsmath}
\usepackage{float}
\usepackage[margin=1.5in]{geometry}
\usepackage{enumitem}
\usepackage{xcolor}
\usepackage{cancel}



\title{Introduction to Mechanics}
\author{Nathan Butcher}

\begin{document}
\newcommand{\scinot}[2]{#1 \cdot 10^{#2}}
\newcounter{chp}
\setcounter{chp}{0}
\newcounter{example}

%Start a new example using counters
\newcommand{\example}{\textbf{Example \texttt{\thechp}.\texttt{\theexample}}
\addtocounter{example}{1}

\hspace{10pt}}

%Give a horizontal line with space to separte things
%like examples or asides
\newcommand{\linespace}{\hspace{10pt}

\hrule

\hspace{10pt}}

\chapter{Introduction, Numbers, and Units}

\setcounter{example}{1}
\addtocounter{chp}{1}

In physics we study the interactions and motion of matter. For our introduction in this class, we will mostly study what I like to call ``Physics at the human scale'' because it is the physics of daily life that we observe and directly interact with. It describes an apple falling from a tree, a car slamming its brakes, a pendulum swinging, the collision of 2 billiard balls, and more.

\section{Math and scientific notation}

This text will assume that you are comfortable with using algebra to solve an equation for a variable and that you have had a class on trigonometry. We will review those topics as they come up but it may be difficult to follow without the math background.

During this text we will use \textbf{scientific notation} frequently to write very large or very small numbers in an easier to read format. With it, we write a number in the format

\begin{equation}
\scinot{a}{b}
\end{equation}

where $1 \leq a < 10$ and $b$ is an integer. This saves us from having to write and count a lot of zeros that indicate place. The number is a factor multiplied by a power of 10.

\linespace

\example

Write the number 38,500,000 in scientific notation.

\hspace{10pt}

To write this in the form $\scinot{a}{b}$, first let's find $a$ such that $1 \leq a < 10$. Since our number starts with the digits 385, that means $a = 3.85$. From here we have

\begin{equation}
38,500,000 = 3.85 \cdot 10,000,000
\end{equation}

Now we need to write 10,000,000 as a power of 10. There are 7 zeros, so it will be $10^7$. This means that our number is 

\begin{equation}
\scinot{3.85}{7}
\end{equation}

\linespace

\example

Write the number 0.000007018 in scientific notation.

\hspace{10pt}

We can use the same reasoning as the above example to find that $a = 7.018$. The number can be written as

\begin{equation}
0.000007018 = 7.018 \cdot 0.000001
\end{equation}

Now we just need to find the power of 10. Let's start looking at powers of 10.

\begin{equation}
\begin{split}
10^0 = 1 \\
10^{-1} = \frac{1}{10} = 0.1 \\
10^{-2} = \frac{1}{10^2} = 0.01 \\
10^{-3} = \frac{1}{10^3} = 0.001
\end{split}
\end{equation}

A negative power means to take 1 divided by the number to that positive power. Looking at 10, we see that as we decrease the power (larger negative value), we move the decimal point 1 to the left. This will give us

\begin{equation}
0.000001 = \frac{1}{10^6} = 10^{-6}
\end{equation}

which makes our number

\begin{equation}
0.000007018 = \scinot{7.018}{-6}
\end{equation}

\section{Units}

Physicists will try to describe the world around us quantitatively, meaning with numbers. In order to ascribe meaning to those numbers we need to use physical \textbf{units}. For example, think about if you were baking a loaf of bread and the recipe called for ``4 flour''. Would you know how much flour was required? The recipe should say ``4 \textit{cups} of flour'' so that you know the amount required for the bread. 

In physics, like most sciences, we use \textbf{SI units}. In the USA, you likely encounter the Imperial system of units in daily life. Common Imperial units you might have seen include the inch for length and gallon for volume. These units are not the standard for scientific disciplines, and as such we will focus on using SI units. There are 3 base units that we will use frequently:

\begin{itemize}
\item \textbf{kilogram} - the SI base unit of mass. Mass is a measure of how much matter there is. It is related to but not the same as weight. We will discuss weight further in a later chapter, but for now think of mass as how much ``stuff'' there is.

\item \textbf{meter} - the SI base unit of length. If you don't have a good mental picture of a meter, it is a little longer than 3 feet.

\item \textbf{second} - the SI base unit of time. We will continue to see minutes and hours used throughout, but for doing quantitative work we will want to use seconds for everything.
\end{itemize}

\section{SI Prefixes}

With these SI base units we use \textbf{prefixes} to scale the units relative to our problem. Table \ref{SIPrefixes} contains several of the most common prefixes, with the bolded ones being most important for this class.

\begin{table}[b]
\centering
\caption{SI Prefixes}
\begin{tabular}{ c | c | c }
	\hline
	Prefix & Abbreviation & Numerical value \\
	\hline
	nano- & n & $10^{-9}$ \\
	micro- & $\mu$ (Greek letter \textit{mu}) & $10^{-6}$ \\
	\textbf{milli-} & m & $10^{-3}$ \\
	\textbf{centi-} & c & $10^{-2}$ \\
	\textbf{kilo-} & k & $10^3$ \\
	mega- & M & $10^6$ \\
	giga- & G & $10^9$ \\
	\hline
\end{tabular}

\label{SIPrefixes}
\end{table}

One thing you should see is that SI prefixes allow us to scale our units by powers of 10. This is much simpler than the Imperial system. For lengths we have 12 inches to a foot, 3 feet to a yard, and 1760 yards to a mile. Compare this to just multiplying or dividing by powers of 10 and see how much simpler SI units are to work with!

The main purpose to having the prefixes is it allows us to scale the units to our problem. If we were to talk about the distance between San Diego and Los Angeles, it would make sense to use kilometers because that distance is about 194 kilometers. In meters, it would be $194 \cdot 10^3 = 194,000$ meters. Prefixes allow us to use numbers that are closer to 1, which are easier to read and for our brains to process. 

There are other prefixes and units that are used within different branches of physics and other fields of science. We will see some of this in CHAPTER HERE when we discuss the solar system. Again, the goal is always to make the numbers we work with closer to 1.

In addition to the base units, we will also have \textbf{derived units}, which are a combination of SI base units. On a car speedometer you may have seen the unit $km/hr$, which means ``kilometers per hour''. This is a derived unit because it combines the units of length and time.


\section{Converting units}

Now that we have SI prefixes, it is important that we learn how to convert between different units. This process is called \textbf{dimensional analysis} because the units are the ``dimensions'' that we are looking at.

To convert units, we will write unit factors that are fractions equal to 1. For example, 

\begin{equation}
\frac{1 \, mm}{10^{-3} \, m} = 1
\end{equation}

is a unit factor that equals 1 because the prefix milli- means $10^{-3}$, so 1 millimeter is equal to $10^{-3}$ meter. This fraction can by multiplied by a quantity to change the unit but not the physical value because the fraction is 1. Now lets use that to convert 3.25 meters to millimeters. We can use the unit fraction to cancel the meters from the original value, and have a unit of millimeters left in the numerator.

\begin{equation}
3.25 \, \cancel{m} \cdot \frac{1 \, mm}{10^{-3} \, \cancel{m}} = 3,250 \, mm
\end{equation}


The reason this works is that our unit fraction is equal to 1, so multiplying our original quantity by that unit fraction doesn't change the value! It only changes what unit that value is given in.

You could also think of the unit fraction as 1,000 millimeters in a meter.

\begin{equation}
3.25 \, \cancel{m} \cdot \frac{1000 \, mm}{1 \, \cancel{m}} = 3,250 \, mm
\end{equation}

Both $\frac{1000 \, mm}{1 \, m}$ and $\frac{1 \, mm}{10^{-3} \, m}$ are valid unit fractions for converting from meters to millieters. They are equally correct, so use whichever you prefer!

\linespace

%\textbf{Example \texttt{\thechp}.\texttt{\theexample}}
%\addtocounter{example}{1}
\example

\textbf{Convert 500 nanometers to centimeters.}

\hspace{10pt}

We can convert 500 nanometers to meters first, then convert from meters to centimeters.

\begin{equation}
500 \, \cancel{nm} \cdot = \frac{10^{-9} \, m}{1 \, \cancel{nm}} = 5 \cdot 10^{-7} \, m
\end{equation}

\begin{equation}
\scinot{5}{-7} \, \cancel{m} = \frac{100 \, cm}{1 \, \cancel{m}} = \scinot{5}{-5} \, cm 
\end{equation}

Sometimes it is useful to convert to an intermediary unit to help you get to the requested final unit.

\linespace

\example

Convert 1 meter to feet. Use the following information:

\begin{itemize}
\item 1 foot = 12 inches
\item 1 inch = 2.54 cm
\end{itemize}

The given unit conversions tell us how to solve this problem. First we convert the meter to centimeters, then use the given conversion from centimeters to inches. Finally, we can use the given conversion from inches to feet.

\begin{equation}
1 \, \cancel{m} \cdot \frac{100 \, \cancel{cm}}{1 \, \cancel{m}} \cdot \frac{1 \, \cancel{in}}{2.54 \, \cancel{cm}} \cdot \frac{1 \, foot}{12 \, \cancel{in}} = 3.28 \, feet
\end{equation}

\linespace

\example

A car is driving at 45 kilometers per hour $\frac{km}{hr}$. Convert this to meters per second $\frac{m}{s}$

\hspace{10pt}

First let's write out the unit to help us see how to convert this.

\begin{equation}
45 \, \frac{km}{hr}
\end{equation}

This means that the car moves 45 kilometers in 1 hour. When doing unit conversions it may be helpful to write out the unit like that

\begin{equation}
45 \, \frac{km}{hr} = \frac{45 \, km}{1 \, hr}
\end{equation}

We have kilometers in the numerator and hours in the denominator, so we have to make sure we write our unit fractions correctly to cancel them. First, let's convert the kilometers to meters.

\begin{equation}
\frac{45 \, \cancel{km}}{1 \, hr} \cdot \frac{10^3 \, m}{1 \, \cancel{km}} = \frac{45,000 \, m}{1 \, hr}
\end{equation}

Now we can convert the hour to seconds. It may be helpful to convert to minutes first, then to seconds.

\begin{equation}
\frac{45,000 \, m}{1 \, \cancel{hr}} \cdot \frac{1 \, \cancel{hr}}{60 \, \cancel{min}} \cdot \frac{1 \, \cancel{min}}{60 \, s} = \frac{12.5 \, m}{1 \, s}
\end{equation}

The car is driving at $12.5 \, \frac{m}{s}$.



\chapter{Quantities of motion and kinematics}
\setcounter{example}{1}
\addtocounter{chp}{1}

The first thing we must do to describe the motion of objects is define specific terms that we will use. These terms will have clear, specific definitions that may differ from how they are used in everyday language. 

The first quantity to talk about is \textbf{position}. This is a measurement of where something is along an axis (or multiple axes). This is a length, so the SI unit for position is the meter. Position will be denoted with the variable $x$ in equations.

When we study motion, we often look at an object moving from one position to the other. The first position is the \textbf{initial} position, and the second position is the \textbf{final} position. In equations we will use subscript $i$ and $f$ for initial final respectively, so that the initial position is $x_i$ and the final position is $x_f$. To indicate the change in a quantity we will use $\Delta$, which is the Greek letter Delta. The change in position between the initial state and final state can be written as

\begin{equation}
\Delta x = x_f - x_i
\end{equation}

which is a measure of how the position changed moving from the initial state $x_i$ to the final state $x_f$. 

The second quantity we will talk about is \textbf{velocity}. This is a measurement of the rate of change in position of a moving object.  Velocity will be denoted with the variable $v$ in equations. Velocity has a direction, so it is a vector quantity. The direction can be denoted with a positive or negative sign to indicate the direction along a position axis.

\textbf{Average velocity} is the average rate of change in position as an object moves from its initial position $x_i$ to its final position $x_f$ over a time interval $\Delta t$. For average velocity we will use a subscript ``avg'', so it is written as $v_{avg}$. 

\begin{equation}
v_{avg} = \frac{\Delta x}{\Delta t}
\end{equation}

Average velocity does not give us any information about the moment-to-moment velocity throughout the motion, only the average throughout the time interval.

From the equation above, we see that velocity is a length divided by time. This means our SI unit is $\frac{m}{s}$ , which is read as \textit{meters per second}. 

The third quantity we will talk about is \textbf{acceleration}. This is a measurement of the rate of change of velocity of a moving object. Acceleration will be denoted with the variable $a$ in equations. Acceleration has a direction, so it is a vector quantity. 

\textbf{Average acceleration} is the average rate of change in velocity as an object moves from its initial position $x_i$ to its final position $x_f$ over a time interval $\Delta t$. Similar to position, the initial velocity is $v_i$, the final velocity is $v_f$, and the change in velocity is $\Delta v = v_f - v_i$.

\begin{equation}
a_{avg} = \frac{\Delta v}{\Delta t}
\end{equation}

Again, average acceleration gives us the average value over the entire interval of motion but no information about the moment-to-moment value of the acceleration.

From the equation above, we see that acceleration is a velocity divided by time. Our SI unit is $\frac{m/s}{s}$, which is read as \textit{meters per second per second}. The unit can also be written as $\frac{m}{s^2}$, which is read as \textit{meters per second squared}. These two ways of writing the units of acceleration are equivalent and both are correct. For the rest of this text, I will use $\frac{m}{s^2}$.

The final quantity we will introduce is \textbf{speed}. This is a measurement of how fast an object is moving. 

\end{document}